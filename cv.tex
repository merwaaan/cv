\documentclass[]{friggeri-cv}

\addbibresource{bibliography.bib}

\usepackage{fancyhdr}

% Add page numbering in the bottom-right corner.
\pagestyle{fancy}
\fancyfoot{}
\fancyfoot[R]{\footnotesize \thepage\ / 2}
\renewcommand{\headrulewidth}{0pt}
\renewcommand{\footrulewidth}{0pt}

\colorlet{lightergray}{lightgray!40}

\begin{document}

\header{merwan}{achibet}
       {Doctorant}

% In the aside, each new line forces a line break
\begin{aside}
  \section{Contact}
    ~
    129 avenue Aristide Briand
    35000 Rennes
    France
    ~
    merwan.achibet@inria.fr
    ~
    06.03.30.56.77
  \section{Web}
    ~
    \href{http://people.irisa.fr/Merwan.Achibet/}{people.irisa.fr/Merwan.Achibet}
    \href{http://github.com/merwaaan}{github.com/merwaaan}
    ~
    {\color{lightergray}\rule{3cm}{0.01cm}}
    ~
    Né le 31/05/1989 (24 ans)
    Nationalité française
    ~
    Détenteur du permis B
\end{aside}

\section{Interêts}

Mondes virtuels, simulation de phénomènes du réel, infographie, web dynamique, systèmes complexes, graphes et réseaux d'interaction.

\section{Formation}

\begin{entrylist}
  \entry
    {2012–2015}
    {Doctorant en réalité virtuelle}
    {Inria Rennes}
    {Manipulation dextre d’objets virtuels avec retour haptique.}
  \entry
    {2010–2012}
    {Master Informatique, mention Très Bien}
    {Université du Havre}
    {Modélisation, Interactions et Systèmes Complexes.}
  \entry
    {2007–2010}
    {Licence Mathématiques-Informatique}
    {Université du Havre}
    {}
  \entry
    {2006–2007}
    {Baccalauréat Scientifique}
    {Lycée Jules Siegfried, Le Havre}
    {Specialités mathématiques et sciences de l'ingénieur.}
\end{entrylist}

\section{Connaissances}

Réalité virtuelle, infographie et OpenGL, interfaces homme-machine, modélisation des systèmes complexes, intelligence artificielle, web dynamique, graphes, algorithmique, programmation orientée objet, programmation fonctionnelle.

\section{Expérience}

\begin{entrylist}
  \entry
    {2013-2014}
    {INSA, Rennes}
    {Vacation}
    {Conception et soutien d'un TD \textit{Modélisation physique pour les applications médicales} pour étudiants de 5\textsuperscript{ème} année d'école d'ingénieur.}
  \entry
    {08–09 2012}
    {LITIS/Université du Havre}
    {CDD}
    {Développement d'un outil d'importation de données géographiques vers des graphes.}
  \entry
    {03–06 2012}
    {LITIS/Université du Havre}
    {Stage de recherche}
    {Coévolution du viaire et du bâti dans un réseau urbain.}
  \entry
    {2010–2012}
    {Université du Havre}
    {Contrats ponctuels}
    {Représentant de la branche scientifique de l'université
      lors de salons étudiants.}
  \entry
    {2010–2011}
    {Université du Havre}
    {Tutorat}
    {Encadrement d'étudiants de licence pour l'étude et la pratique du
    C.}
  \entry
    {04–07 2010}
    {BNP Paribas, Paris}
    {Stage de licence + CDD}
    {Réalisation d'un outil de gestion d'expertises immobilières.}
  \entry
    {07–08 2009}
    {Groupama Transport, Le Havre}
    {CDD}
    {Maintenance, dépannage et manutention informatique.}
  \entry
    {07–08 2008}
    {Groupama Transport, Le Havre}
    {CDD}
    {Maintenance, dépannage et manutention informatique.}
\end{entrylist}

\newpage

\section{Langues}

\begin{entrylist}
  \entry
    {}
    {Français}
    {}
    {Langue maternelle.}
  \entry
    {}
    {Anglais}
    {}
    {Courant, score de 940/990 au TOEIC.}
\end{entrylist}

\section{Autres langues}

\begin{entrylist}
  \entry
    {}
    {Impératives}
    {}
    {C, C++, Python, Java.}
  \entry
    {}
    {Fonctionnelles}
    {}
    {Scheme, Common Lisp, OCaml.}
  \entry
    {}
    {Logique}
    {}
    {Prolog.}
  \entry
    {}
    {Orientées web}
    {}
    {Javascript, Coffeescript, HTML5, CSS3, PHP, SQL.}
\end{entrylist}

\section{Publications}

\begin{entrylist}
  \entry
    {2014}
    {The Virtual Mitten: A Novel Interaction Paradigm for Visuo-Haptic\\ Manipulation of Objects Using Grip Force}
    {}
    {Merwan Achibet, Maud Marchal, Ferran Argelaguet, Anatole Lécuyer.\\ IEEE 3DUI. Minneapolis, Etats-Unis.}
  \entry
    {}
    {A Model of Road Network and Buildings Extension Co-evolution}
    {}
    {Merwan Achibet, Stefan Balev, Antoine Dutot, Damien Olivier.\\ AgentCities workshop within ANT. Hasselt, Belgique.}
  \entry
    {2012}
    {Co-Evolution of the Road Network and the Land Use in a City}
    {}
    {Merwan Achibet, Stefan Balev, Antoine Dutot, Damien Olivier.\\ SOMC workshop within ECCS, Bruxelles, Belgique.}

\end{entrylist}

\section{Projets}

\begin{entrylist}
  \entry
    {2012}
    {Sparkets}
    {\href{http://github.com/fmdkdd/sparkets}{github.com/fmdkdd/sparkets}}
    {Jeu vidéo multijoueur temps-réel par navigateur tirant partie des
      dernières technologies du web : HTML5, Node.js et Socket.IO.}
  \entry
    {2011}
    {Simulation physique de corps rigide avec interaction}
    {\href{http://github.com/merwaaan/physics}{github.com/merwaaan/physics}}
    {Projet de master étudiant la simulation de l'évolution des
      propriétés mécaniques d'objets rigides dans un environnement 3D
      en prenant en compte les interactions collisionnelles s'opérant
      entre les corps et s'étant conclu par la mise au point d'un
      moteur physique.}
  \entry
    {2010}
    {La programmation et l'art}
    {\href{http://github.com/merwaaan/papers/blob/master/art/progart.pdf?raw=true}{github.com/merwaaan/papers/art}}
    {Dissertation réalisée en licence à propos de la relation
      art/programmation ou comment un concept intrinsèquement humain
      peut être appliqué à un support purement synthétique.}

\end{entrylist}

\end{document}
