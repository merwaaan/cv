%!TEX TS-program = xelatex
\documentclass[]{friggeri-cv}

\begin{document}
\header{merwan}{achibet}
       {}

% In the aside, each new line forces a line break
\begin{aside}
  \section{à propos}
    31 rue Toulouse Lautrec
    76620 Le Havre
    France
    ~
    \href{http://merwaaan.github.com}{merwaaan.github.com}
    ~
    \href{mailto:achibetmerwan@gmail.com}{achibetmerwan@gmail.com}
    +33 6 03 30 56 77
  \section{langues}
    Français
    Anglais
  \section{programmation}
    JavaScript
    HTML5
    CSS3
    ~
    C
    Java
    C++
    ~
    Common Lisp
    Prolog
\end{aside}

\section{interêts}

Simulation de phénomènes du réel, intelligence artificielle,

\section{éducation}

\begin{entrylist}
  \entry
    {depuis 2010}
    {Master Informatique}
    {Université du havre}
    {\textit{Modélisation, Interactions et Systèmes Complexes}}
  \entry
    {2007–2010}
    {Licence Mathématiques-Informatique}
    {Université du Havre}
    {}
  \entry
    {2007}
    {Baccalauréat Scientifique}
    {Lycée Jules Siegfried, Le Havre}
    {Specialités mathématiques et sciences de l'ingénieur}
\end{entrylist}

\section{expérience}

\begin{entrylist}
  \entry
    {actuellement}
    {LITIS/Université du Havre}
    {Stage de recherche}
    {Coévolution du viaire et du bâti dans un réseau urbain.}
  \entry
    {06–07 2010}
    {BNP Paribas, Paris}
    {Job d'été}
    {Continuité du stage de licence.}
  \entry
    {04–05 2010}
    {BNP Paribas, Paris}
    {Stage}
    {Mise au point d'un outil de gestion d'expertises immobilières.}
  \entry
    {07–08 2009}
    {Groupama Transport, Le Havre}
    {Job d'été}
    {Maintenance, dépannage et manutention informatique.}
  \entry
    {07–08 2008}
    {Groupama Transport, Le Havre}
    {Job d'été}
    {Maintenance, dépannage et manutention informatique.}
\end{entrylist}

\section{projets}

\begin{entrylist}
  \entry
    {2012}
    {\href{http://github.com/fmdkdd/sparkets}{Sparkets}}
    {}
    {Jeu vidéo multijoueur par navigateur tirant partie des dernières
      technologies du web : HTML5, Nodejs et Socket.IO.}
  \entry
    {2011}
    {\href{http://github.com/merwaaan/physics}{Simulation physique de corps rigide avec interaction}}
    {}
    {Projet de master étudiant la simulation de l'évolution des
      propriétés mécaniques d'objets rigides en prenant en compte les
      interactions collisionnelles s'opérant entre les corps et
      s'étant conclu par la mise au point d'un moteur physique.}
  \entry
    {2010}
    {\href{http://github.com/merwaaan/papers/blob/master/art/progart.pdf?raw=true}{La programmation et l'art}}
    {}
    {Dissertion réalisée en licence à propos de la relation
      art/programmation ou comment un concept intrinsèquement humain
      peut-il être appliqué à un support purement synthétique.}

\end{entrylist}

\end{document}
