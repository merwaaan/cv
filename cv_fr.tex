\documentclass[]{friggeri-cv}

% Add page numbering in the bottom-right corner.
\usepackage{fancyhdr}
\pagestyle{fancy}
\fancyfoot{}
\fancyfoot[R]{\footnotesize \thepage\ / 2}
\renewcommand{\headrulewidth}{0pt}
\renewcommand{\footrulewidth}{0pt}

\colorlet{lightergray}{lightgray!40}

\begin{document}

\header{merwan}{achibet}
       {Doctorant en Réalité Virtuelle}
			
\begin{aside}
  \section{à propos}\vspace{.25cm}
    129 avenue Aristide Briand
    35000 Rennes
    France
    ~
    \href{mailto:merwan.achibet@inria.fr}{merwan.achibet@inria.fr}
    +33 6 0330 5677
		~
    \href{http://people.irisa.fr/Merwan.Achibet/}{people.irisa.fr/Merwan.Achibet}
    \href{http://github.com/merwaaan}{github.com/merwaaan}
    {\color{lightergray}\rule{3cm}{0.01cm}}
	\section{languages}\vspace{.25cm}
		Français {\footnotesize -- Langue maternelle}\vspace{.4cm}
		Anglais {\footnotesize -- ???}
		{\footnotesize (TOEIC 940/990)}\vspace{.4cm}
		Japonais {\footnotesize -- Notions}
\end{aside}

\section{Interêts}

Mondes virtuels, Simulation, Développement de jeux, Infographie et rendu, Web dynamique.

\section{Formation}

\begin{entrylist}
  \entry
    {2012–2015}
    {Doctorant en réalité virtuelle}
    {Inria Rennes}
    {\it Manipulation dextre d’objets virtuels avec retour haptique.}
  \entry
    {2010–2012}
    {Master Informatique, mention Très Bien}
    {Université du Havre}
    {Modélisation, Interactions et Systèmes Complexes.}
  \entry
    {2007–2010}
    {Licence Mathématiques-Informatique}
    {Université du Havre}
    {}
\end{entrylist}

\section{Compétences}

\begin{entrylist}
	\entry
		{Code}
		{C\#, C++, Python, Java}
		{}
		{}
	\entry
		{Web \& mobile}
		{Javascript, CoffeeScript, HTML5, Android SDK}
		{}
		{}
	\entry
		{Outils}
		{Unity3D, OpenGL, Bullet, Blender, Qt Creator, Visual Studio, Node.js, Git}
		{}
		{}
	\entry
		{Hardware}
		{CasqueVR headsets, 3D user interfaces, Motion capture, Haptic devices, Arduino}
		{}
		{}
\end{entrylist}

\section{Publications}

\begin{entrylist}
  \entry
    {2015}
    {THING: Introducing a Tablet-based Interaction Technique\\ for Controlling 3D Hand Models}
    {Séoul, Corée du Sud}
    {M. Achibet, G. Casiez, A. Lécuyer, M. Marchal.\\ ACM CHI.}
  \entry
    {}
    {Elastic-Arm: Human-Scale Haptic Feedback for Augmenting\\ Interaction and Perception in Virtual Environments}
    {Arles, France}
    {M.Achibet, A. Gérard, A. Talvas, M. Marchal, A. Lécuyer.\\ IEEE VR.}
  \entry
    {2014}
    {The Virtual Mitten: A Novel Interaction Paradigm for\\ Visuo-Haptic Manipulation of Objects Using Grip Force}
    {Minneapolis, \'Etats-Unis}
    {M. Achibet, M. Marchal, F. Argelaguet, A. Lécuyer.\\ IEEE 3DUI, \underline{Prix du meilleur article}.}
  \entry
    {}
    {A Model of Road Network and Buildings Extension Co-evolution}
    {Hasselt, Belgique}
    {M. Achibet, S. Balev, A. Dutot, D. Olivier.\\ AgentCities workshop within ANT.}
  \entry
    {2012}
    {Co-Evolution of the Road Network and the Land Use in a City}
    {Brussels, Belgique}
    {M. Achibet, S. Balev, A. Dutot, D. Olivier.\\ SOMC workshop within ECCS.}
\end{entrylist}

\newpage

\section{Enseignements}

\begin{entrylist}
  \entry
    {2015}
    {INSA, Rennes \textemdash\ \'Ecole d'ingénieurs}
    {Chargé d'enseignements}
    {Introduction à l'algorithmique et à la programmation orientée objet.}
  \entry
    {2014}
    {INSA, Rennes \textemdash\ \'Ecole d'ingénieurs}
    {Assistance et conception de TD}
    {Simulation physique pour applications médicales.}
  \entry
    {2010–2011}
    {Université du Havre}
    {Tutorat}
    {Le Langage C.}
\end{entrylist}

\section{Experience}

\begin{entrylist}
  \entry
    {08–09 2012}
    {LITIS Lab/Université du Havre}
    {Ingénieur de recherche}
    {Développement d'un outil de conversion de données géographiques en graphes.}
  \entry
    {03–06 2012}
    {LITIS lab/Université du Havre}
    {Stage de recherche}
    {Simulation de l'évolution des parcelles et des routes dans la ville.}
  \entry
    {04–07 2010}
    {BNP Paribas, Paris}
    {Ingénieur}
    {Développement d'un outil de back-end pour agents immobiliers.}
  \entry
    {07–08 2009}
    {Groupama Transport, Le Havre}
    {Technicien informatique}
    {Maintenance et dépannage informatique}
  \entry
    {07–08 2008}
    {Groupama Transport, Le Havre}
    {Technicien informatique}
    {Maintenance et dépannage informatique}
\end{entrylist}

\section{Quelques projets personnels}

\begin{entrylist}
  \entry
    {2015}
    {You are the münster}
    {\href{http://github.com/fmdkdd/yatm}{github.com/fmdkdd/yatm}}
    {Un jeu de plate-forme réalisé en 72 heures pour la game jam \textit{Ludum Dare} en équipe de deux.}
  \entry
    {}
    {Cheapo.js}
    {\href{http://github.com/merwaaan/cheapo.js}{github.com/merwaaan/cheapo.js}}
    {Un interpréteur de jeux CHIP-8 et Super CHIP-8 écrit en CoffeeScript.}
  \entry
    {2014}
    {Boyo.js}
    {\href{http://github.com/merwaaan/boyo.js}{github.com/merwaaan/boyo.js}}
    {Un émulateur de Game Boy écrit en JavaScript pour fonctionner dans un navigateur.}
  \entry
    {2013}
    {Sparkets}
    {\href{http://github.com/fmdkdd/sparkets}{github.com/fmdkdd/sparkets}}
    {Un jeu de bataille spatial multijoueur jouable dans le navigateur.}
\end{entrylist}

\end{document}
